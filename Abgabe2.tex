% Definierung des Layouts. Article wird fuer wissenschaftliche Texte genutzt
\documentclass{article}
% Definierung in welcher Kodierung die Datei gelesen werden soll (sonst werden Umlaute nicht unerstuetzt)
\usepackage[utf8]{inputenc}
% Korrekte Silbentrennung im Deutschen
\usepackage[ngerman]{babel}

% Java code Formatierung Anfang
\usepackage{listings}
\usepackage{color}
\definecolor{name}{rgb}{0.5,0.5,0.5}
\definecolor{javared}{rgb}{0.6,0,0}
\definecolor{javagreen}{rgb}{0.25,0.5,0.35}
\definecolor{javapurple}{rgb}{0.5,0,0.35}
\definecolor{javadocblue}{rgb}{0.25,0.35,0.75}
\DeclareUnicodeCharacter{00A0}{~}
\lstset{language=Java,
basicstyle=\ttfamily,
keywordstyle=\color{javapurple}\bfseries,
stringstyle=\color{javared},
commentstyle=\color{javagreen},
morecomment=[s][\color{javadocblue}]{/**}{*/},
numbers=left,
numberstyle=\tiny\color{black},
stepnumber=1,
numbersep=10pt,
tabsize=4,
showspaces=false,
showstringspaces=false}
% Java code Formatierung Ende


% Importierung der Mathe Zeichen Anfang
\usepackage[sumlimits,intlimits,namelimits]{amsmath}
\usepackage{amssymb}
\usepackage{pifont}
\usepackage{graphicx} 
\usepackage{verbatim}
\usepackage{epsf}
\newcommand{\R}{\mbox{I}\!\mbox{R}}
\newcommand{\N}{\mbox{I}\!\mbox{N}}
\newcommand{\Rn}{\mathbb{R}}
\newcommand{\C}{\mathbb{C}}
\DeclareMathOperator{\Div}{div}
 % Importierung der Mathe Zeichen Ende


% Definierung von Titel Autor und Datum Anfang
\title{Datenstrukturen und Algorithmen, Abgabe 2}
\author{Beccard, Vincent, 377979; Braun, Basile 388542; Brungs, Simon, 377281; Nummer der Übungsgruppe: 22}
\date{\today}
% Definierung von Titel Autor und Datum Ende

%Definierung von Kopfzeile und Fuszzeile Anfang
\usepackage{scrpage2}
\pagestyle{scrheadings}
\clearscrheadfoot
\chead{Beccard, Vincent, 377979; Braun, Basile 388542; Brungs, Simon, 377281; Nummer der Übungsgruppe: 22}
\cfoot[\pagemark]{\pagemark}
%Definierung von Kopfzeile und Fuszzeile Anfang

%Der Anfang des eigentlichen Dokuments
\begin{document}
%Erstellung der Kopfzeile des ersten Dokumentes
\noindent
Gruppennummer: 22%% \hspace{\fill} right-hand text
\begingroup
\let\newpage\relax% Void the actions of \newpage
% Generierung des Titels
\maketitle
\endgroup
%Dieses Kommando wird genutzt damit Subsections mit a und b anfangen anstatt mit 1.1 und 1.2
\renewcommand{\thesubsection}{\alph{subsection}}
% Beginn der Importierung von den Lösungen der Aufgaben Dateiein
\section{Funktionen sortieren}
\subsection{}
\subsubsection{Zeige das $\subseteq{}$ transitiv ist}
Es gilt $f\in O(g) \wedge g \in O(f)\Rightarrow f\in O(h)$\\
Laut Definition von $\subseteq{}$ gilt dementsprechend dann\\
$f\subseteq{}g \wedge g \subseteq{}h \Rightarrow f \subseteq{}h \Leftrightarrow f \Leftarrow O(g) \wedge g \in O(h) \Rightarrow f \in O(h)$\\
Damit ist die Transitivität gezeigt.
\subsubsection{Zeige das$\subseteq{}$  reflexiv ist}
Da $f \in O(f)$ gilt folgt laut Definition von $\subseteq{}$ das auch $f\subseteq{}f$ 
Damit ist die Reflexivität gezeigt.
\subsection{}
0$\subseteq 4 \subseteq 2^9000 \subseteq 2 \subseteq \log{n} \subseteq n \sqrt{n} \subseteq n^2 \subseteq n^2 \cdot \log{n} \subseteq \sum\nolimits_{i=0}^n \frac{14i^2}{1+i} \subseteq \frac{n^3}{2} \subseteq n^3 \subseteq 2^n \subseteq n! \subseteq n^n$ 
\section{O-Notation konkret}
\subsection{Behauptung: $\frac{1}{4}n^3-7n+17 \in O (n^3)$}
$\limsup\limits_{n \to \infty} \frac{1}{4}n^3-7n+17 \in O (n^3)= \frac{1}{4} \geq 0$, $\frac{1}{4} \neq \infty$\\
Damit ist die Behauptung bewiesen. \hspace{2cm} $\Box$
\subsection{Behauptung: $n^4 \in O(2n^4+3n^2+42)$}
$\limsup\limits_{n \to \infty} n^4 \in O(2n^4+3n^2+42)$ (L'Hôpital)\\
\hspace*{44mm}$=\limsup\limits_{n \to \infty} \frac{24n^3}{48n^2}= \frac{1}{2} \geq 0$\\
Damit ist $n^4 \in 0 (2n^4 + 3n^2+ 42)$ bewiesen. \hspace{2cm} $\Box$
\section{Laufzeitanalyse}
\subsection{Bestimmung in Abhängigkeit von n die Best-case Laufzeit $B(n)$}
Die Best-case Laufzeit ist $B(n)=1+2n$.
\subsection{Bestimmung in Abhängigkeit von n die Worst-case Laufzeit $W(n)$}
Die Worst-case Laufzeit ist $W(n)=1+3n$.
\subsection{Bestimmung in Abhängigkeit von n der Average-case Laufzeit $A(n)$}
Die Average-case Laufzeit ist $A(n)=\sum\nolimits_{i=0}^n \frac{n!}{i!} \cdot 0.5^2 \cdot (1-0.5)^{n-i}$
\subsection{Äquivalenter Algorithmus mit besserer Average-case Laufzeit}
Der folgende äquivalenter Algorithmus hat eine besser Average-case Laufzeit ab $n>2$, da er nur einmal am Ende abfragt, ob m gleich 0 ist, anstatt bei jedem Element des Arrays, welches true ist. Somit spart man für jedes Element welches true enthält einen Durchlauf und nur im relativ unwahrscheinlichen Fall, wenn alle Elemente im Array false sind braucht er einen mehr.
\lstinputlisting{BessereLaufzeit.java}
\subsection{Äquivalenter Algorithmus, dessen Worst-case Laufzeit in o(n) liegt}
Nein, gibt es nicht
% Beginn der Importierung von den Lösungen der Aufgaben Dateiein
\end{document}
%Beendigung des eigentlichen Dokuments