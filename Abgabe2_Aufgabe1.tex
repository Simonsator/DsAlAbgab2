\section{Funktionen sortieren}
\subsection{}
\subsubsection{Zeige das $\subseteq{}$ transitiv ist}
Es gilt $f\in O(g) \wedge g \in O(f)\Rightarrow f\in O(h)$\\
Laut Definition von $\subseteq{}$ gilt dementsprechend dann\\
$f\subseteq{}g \wedge g \subseteq{}h \Rightarrow f \subseteq{}h \Leftrightarrow f \Leftarrow O(g) \wedge g \in O(h) \Rightarrow f \in O(h)$\\
Damit ist die Transitivität gezeigt.
\subsubsection{Zeige das$\subseteq{}$  reflexiv ist}
Da $f \in O(f)$ gilt folgt laut Definition von $\subseteq{}$ das auch $f\subseteq{}f$ 
Damit ist die Reflexivität gezeigt.
\subsection{}
0$\subseteq 4 \subseteq 2^9000 \subseteq 2 \subseteq \log{n} \subseteq n \sqrt{n} \subseteq n^2 \subseteq n^2 \cdot \log{n} \subseteq \sum\nolimits_{i=0}^n \frac{14i^2}{1+i} \subseteq \frac{n^3}{2} \subseteq n^3 \subseteq 2^n \subseteq n! \subseteq n^n$ 