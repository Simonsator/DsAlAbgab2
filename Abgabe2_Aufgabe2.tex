\section{O-Notation konkret}
\subsection{Behauptung: $\frac{1}{4}n^3-7n+17 \in O (n^3)$}
$\limsup\limits_{n \to \infty} \frac{1}{4}n^3-7n+17 \in O (n^3)= \frac{1}{4} \geq 0$, $\frac{1}{4} \neq \infty$\\
Damit ist die Behauptung bewiesen. \hspace{2cm} $\Box$
\subsection{Behauptung: $n^4 \in O(2n^4+3n^2+42)$}
$\limsup\limits_{n \to \infty} n^4 \in O(2n^4+3n^2+42)$ (L'Hôpital)\\
\hspace*{44mm}$=\limsup\limits_{n \to \infty} \frac{24n^3}{48n^2}= \frac{1}{2} \geq 0$\\
Damit ist $n^4 \in 0 (2n^4 + 3n^2+ 42)$ bewiesen. \hspace{2cm} $\Box$